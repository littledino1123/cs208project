\documentclass{article}
\input{lib.sty}

\title{\texttt{fn make\_lipschitz\_float\_mul}}
\author{David Erb}
\date{}
\lstdefinelanguage{Rust}{
  morekeywords={let,mut,fn,enum,impl,match,if,else,loop,while,for,in,break,continue,return,struct,trait,self,Self,mod,ref,move,crate,super,where,use,as,const,static,unsafe,extern,dyn,await,async},
  sensitive=true,
  morecomment=[l]{//},
  morecomment=[s]{/*}{*/},
  morestring=[b]",
}


\begin{document}

\maketitle

Proves soundness of \texttt{fn make\_lipschitz\_float\_mul}

%\subsection*{Vetting History}
%\begin{itemize}
%    \item \vettingPR{512}
%end{itemize}

\section{Hoare Triple}
\subsection*{Precondition}
To ensure the correctness of the output, we require the following preconditions:

\begin{itemize}
    \item Type \texttt{TA} must have Datatype \texttt{Float}.
\end{itemize}

\subsection*{Pseudocode}
\lstinputlisting[language=Rust]{./pseudocode/make_lipschitz_mul.rs}

\subsubsection*{Postconditions}
\validTransformation{\texttt{(input\_domain, input\_metric, constant, bounds)}}{\texttt{make\_lipschitz\_float\_mul}}

\section{Proof}

\begin{proof}
    \textbf{(Part 1 - appropriate output domain).}
    Conditioning on the correct implementation of \texttt{saturating\_mul}, the output is going to be in \texttt{AtomDomain::new\_non\_nan()} as long as both \texttt{constant} and the return value of \texttt{total\_clamp( lower, upper)?} are in \texttt{AtomDomain::new\_non\_nan()}.
    
    \texttt{constant} is of type \texttt{TA} which is checked to be not nan (line TODO). Similarly, the bounds as well as \texttt{arg} are non-nan. Thus, by the definition of ProductOrd, the preconditions of \texttt{total\_clamp} are satisfied, and therefore the output is non-nan. Thus, for all settings of input arguments, the function returns a dataset in the ouput domain.
\end{proof}

\begin{proof}
    \textbf{(Part 2 - stability map).} Take any two elements $u,v$ in the \texttt{input\_domain} as bounds $L,U$ and any pair \texttt{(d\_in, d\_out)}. Assume $u,v$ are \texttt{d\_in}-close under the \texttt{input\_metric} and that \texttt{stability\_map(d\_in)} $\le$ \texttt{d\_out}. Let \texttt{IM = input\_metric, OM = output\_metric}.

\begin{align*}
  d_{OM}(f(u),f(v))
  &= \bigl|f(u)-f(v)\bigr|
    &&\text{since \texttt{OM} is \texttt{AbsoluteDistance}}\\
  &= \Bigl|\mathrm{fl}\bigl(c\,\min\{U,\max\{L,u\}\}\bigr)
            -\;\mathrm{fl}\bigl(c\,\min\{U,\max\{L,v\}\}\bigr)\Bigr|\\
  &= \Bigl|\,c\,\min\{U,\max\{L,u\}\} + e
           \;-\;\bigl(c\,\min\{U,\max\{L,v\}\}+e'\bigr)\Bigr|\\
  &= \Bigl|\,c\bigl(\min\{U,\max\{L,u\}\}-\min\{U,\max\{L,v\}\}\bigr)
           \;+\;(e - e')\Bigr|\\
  &\le |c|\;\Bigl|\min\{U,\max\{L,u\}\}-\min\{U,\max\{L,v\}\}\Bigr|
       \;+\;|e|+|e'|
    &&\text{by the triangle inequality}\\
  &\le |c|\;|u - v|
       \;+\;|e|+|e'|
    &&\text{since }\min\{U,\max\{L,\,\cdot\,\}\}\text{ is 1-Lipschitz}\\
  &\le |c|\,d_{\mathrm{in}}
       \;+\;2\,\max(|e|,|e'|)
    &&\bigl|u-v\bigr|\le d_{\mathrm{in}}\\
  &\le |c|\,d_{\mathrm{in}}
       \;+\;\mathrm{ULP}(w)
    &&|e|,|e'|\le\tfrac12\,\mathrm{ULP}(w),\;w=\max(|L|,|U|)\,|c|\\
  &= \texttt{stability\_map}(d_{\mathrm{in}})
    &&\text{by line \ref{line:stability-map} of the pseudocode}\\
  &\le d_{\mathrm{out}}
    &&\text{by the second assumption.}
\end{align*}

\end{proof}
\end{document}