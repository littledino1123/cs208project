\documentclass{article}
\input{lib.sty}

\title{\texttt{fn make\_mean}}
\author{David Erb}
\date{}
\lstdefinelanguage{Rust}{
  morekeywords={let,mut,fn,enum,impl,match,if,else,loop,while,for,in,break,continue,return,struct,trait,self,Self,mod,ref,move,crate,super,where,use,as,const,static,unsafe,extern,dyn,await,async},
  sensitive=true,
  morecomment=[l]{//},
  morecomment=[s]{/*}{*/},
  morestring=[b]",
}


\begin{document}

\maketitle

Proves soundness of \texttt{fn make\_mean}

%\subsection*{Vetting History}
%\begin{itemize}
%    \item \vettingPR{512}
%end{itemize}

\section{Hoare Triple}
\subsection*{Precondition}
To ensure the correctness of the output, we require the following preconditions:

\begin{itemize}
    \item \texttt{MI} is a type with trait \rustdoc{transformations/trait}{Metric}
    \item \texttt{T} is a type with traits \rustdoc{transformations/trait}{MakeSum}\texttt{<MI>}, \rustdoc{transformations/trait}{ExactIntCast}\texttt{<usize>}, \rustdoc{transformations/trait}{Float} and \rustdoc{transformations/trait}{InfMul}
    \item \rustdoc{core/trait}{MetricSpace} is implemented for \texttt{(VectorDomain < AtomDomain<T> >, MI)}
    \item \rustdoc{core/trait}{MetricSpace} is implemented for \texttt{(AtomDomain<T>, AbsoluteDistance<T>)}
\end{itemize}

\subsection*{Pseudocode}
\lstinputlisting[language=Rust]{./pseudocode/make_mean.rs}

\subsubsection*{Postconditions}
\validTransformation{\texttt{(input\_domain, input\_metric)}}{\texttt{make\_mean}}

\section{Proof}
Before proving the postcondistions, we first provide two lemmas.
\begin{lemma} \label{sum}
The preconditions of \texttt{make\_sum} are satisfied.
\end{lemma}
\begin{proof}
    By our assumptions,
    \begin{enumerate}
        \item \texttt{MI} has trait \texttt{Metric}
        \item \texttt{T} has trait \texttt{MakeSum<MI>}
        \item \texttt{MetricSpace} is implemented for \texttt{(VectorDomain < AtomDomain<T> >, MI)}
        \item \texttt{MetricSpace} is implemented for \texttt{(AtomDomain<T>, AbsoluteDistance<T>)}
    \end{enumerate} 
\end{proof}

\begin{lemma} \label{mul}
    The precondition of \texttt{make\_lipschitz\_float\_mul} is satisfied.
\end{lemma}

\begin{proof}
    The inputdomain and inputmetric is going to be filled by the output of \texttt{make\_sum}. By \ref{sum}, its preconditions are fulfilled and the output domain is \texttt{AtomDomain<T>} and the output metric is \texttt{AbsoluteDistance<T>}. By assumption, \texttt{T} has trait \texttt{Float} and therefore the precondition is satisfied.
\end{proof}

We proceed by proving the postconditions.

\begin{proof}
    \textbf{(Part 1 - appropriate output domain).}
    By \ref{sum}, \ref{mul} and the definition of \texttt{make\_lipschitz\_float\_mul} the output domain is \texttt{AtomDomain<T>}.
\end{proof}

\begin{proof}
    \textbf{(Part 2 - stability map).} 
    The stability map of the returned transformation is the composition of the stability maps of the chained transformation. By \ref{sum} and \ref{mul}, the stability guarantee follows immediately.
\end{proof}

\end{document}